% Options for packages loaded elsewhere
\PassOptionsToPackage{unicode}{hyperref}
\PassOptionsToPackage{hyphens}{url}
%
\documentclass[
]{article}
\title{Homework-1}
\author{}
\date{\vspace{-2.5em}}

\usepackage{amsmath,amssymb}
\usepackage{lmodern}
\usepackage{iftex}
\ifPDFTeX
  \usepackage[T1]{fontenc}
  \usepackage[utf8]{inputenc}
  \usepackage{textcomp} % provide euro and other symbols
\else % if luatex or xetex
  \usepackage{unicode-math}
  \defaultfontfeatures{Scale=MatchLowercase}
  \defaultfontfeatures[\rmfamily]{Ligatures=TeX,Scale=1}
\fi
% Use upquote if available, for straight quotes in verbatim environments
\IfFileExists{upquote.sty}{\usepackage{upquote}}{}
\IfFileExists{microtype.sty}{% use microtype if available
  \usepackage[]{microtype}
  \UseMicrotypeSet[protrusion]{basicmath} % disable protrusion for tt fonts
}{}
\makeatletter
\@ifundefined{KOMAClassName}{% if non-KOMA class
  \IfFileExists{parskip.sty}{%
    \usepackage{parskip}
  }{% else
    \setlength{\parindent}{0pt}
    \setlength{\parskip}{6pt plus 2pt minus 1pt}}
}{% if KOMA class
  \KOMAoptions{parskip=half}}
\makeatother
\usepackage{xcolor}
\IfFileExists{xurl.sty}{\usepackage{xurl}}{} % add URL line breaks if available
\IfFileExists{bookmark.sty}{\usepackage{bookmark}}{\usepackage{hyperref}}
\hypersetup{
  pdftitle={Homework-1},
  hidelinks,
  pdfcreator={LaTeX via pandoc}}
\urlstyle{same} % disable monospaced font for URLs
\usepackage[margin=1in]{geometry}
\usepackage{color}
\usepackage{fancyvrb}
\newcommand{\VerbBar}{|}
\newcommand{\VERB}{\Verb[commandchars=\\\{\}]}
\DefineVerbatimEnvironment{Highlighting}{Verbatim}{commandchars=\\\{\}}
% Add ',fontsize=\small' for more characters per line
\usepackage{framed}
\definecolor{shadecolor}{RGB}{248,248,248}
\newenvironment{Shaded}{\begin{snugshade}}{\end{snugshade}}
\newcommand{\AlertTok}[1]{\textcolor[rgb]{0.94,0.16,0.16}{#1}}
\newcommand{\AnnotationTok}[1]{\textcolor[rgb]{0.56,0.35,0.01}{\textbf{\textit{#1}}}}
\newcommand{\AttributeTok}[1]{\textcolor[rgb]{0.77,0.63,0.00}{#1}}
\newcommand{\BaseNTok}[1]{\textcolor[rgb]{0.00,0.00,0.81}{#1}}
\newcommand{\BuiltInTok}[1]{#1}
\newcommand{\CharTok}[1]{\textcolor[rgb]{0.31,0.60,0.02}{#1}}
\newcommand{\CommentTok}[1]{\textcolor[rgb]{0.56,0.35,0.01}{\textit{#1}}}
\newcommand{\CommentVarTok}[1]{\textcolor[rgb]{0.56,0.35,0.01}{\textbf{\textit{#1}}}}
\newcommand{\ConstantTok}[1]{\textcolor[rgb]{0.00,0.00,0.00}{#1}}
\newcommand{\ControlFlowTok}[1]{\textcolor[rgb]{0.13,0.29,0.53}{\textbf{#1}}}
\newcommand{\DataTypeTok}[1]{\textcolor[rgb]{0.13,0.29,0.53}{#1}}
\newcommand{\DecValTok}[1]{\textcolor[rgb]{0.00,0.00,0.81}{#1}}
\newcommand{\DocumentationTok}[1]{\textcolor[rgb]{0.56,0.35,0.01}{\textbf{\textit{#1}}}}
\newcommand{\ErrorTok}[1]{\textcolor[rgb]{0.64,0.00,0.00}{\textbf{#1}}}
\newcommand{\ExtensionTok}[1]{#1}
\newcommand{\FloatTok}[1]{\textcolor[rgb]{0.00,0.00,0.81}{#1}}
\newcommand{\FunctionTok}[1]{\textcolor[rgb]{0.00,0.00,0.00}{#1}}
\newcommand{\ImportTok}[1]{#1}
\newcommand{\InformationTok}[1]{\textcolor[rgb]{0.56,0.35,0.01}{\textbf{\textit{#1}}}}
\newcommand{\KeywordTok}[1]{\textcolor[rgb]{0.13,0.29,0.53}{\textbf{#1}}}
\newcommand{\NormalTok}[1]{#1}
\newcommand{\OperatorTok}[1]{\textcolor[rgb]{0.81,0.36,0.00}{\textbf{#1}}}
\newcommand{\OtherTok}[1]{\textcolor[rgb]{0.56,0.35,0.01}{#1}}
\newcommand{\PreprocessorTok}[1]{\textcolor[rgb]{0.56,0.35,0.01}{\textit{#1}}}
\newcommand{\RegionMarkerTok}[1]{#1}
\newcommand{\SpecialCharTok}[1]{\textcolor[rgb]{0.00,0.00,0.00}{#1}}
\newcommand{\SpecialStringTok}[1]{\textcolor[rgb]{0.31,0.60,0.02}{#1}}
\newcommand{\StringTok}[1]{\textcolor[rgb]{0.31,0.60,0.02}{#1}}
\newcommand{\VariableTok}[1]{\textcolor[rgb]{0.00,0.00,0.00}{#1}}
\newcommand{\VerbatimStringTok}[1]{\textcolor[rgb]{0.31,0.60,0.02}{#1}}
\newcommand{\WarningTok}[1]{\textcolor[rgb]{0.56,0.35,0.01}{\textbf{\textit{#1}}}}
\usepackage{graphicx}
\makeatletter
\def\maxwidth{\ifdim\Gin@nat@width>\linewidth\linewidth\else\Gin@nat@width\fi}
\def\maxheight{\ifdim\Gin@nat@height>\textheight\textheight\else\Gin@nat@height\fi}
\makeatother
% Scale images if necessary, so that they will not overflow the page
% margins by default, and it is still possible to overwrite the defaults
% using explicit options in \includegraphics[width, height, ...]{}
\setkeys{Gin}{width=\maxwidth,height=\maxheight,keepaspectratio}
% Set default figure placement to htbp
\makeatletter
\def\fps@figure{htbp}
\makeatother
\setlength{\emergencystretch}{3em} % prevent overfull lines
\providecommand{\tightlist}{%
  \setlength{\itemsep}{0pt}\setlength{\parskip}{0pt}}
\setcounter{secnumdepth}{-\maxdimen} % remove section numbering
\ifLuaTeX
  \usepackage{selnolig}  % disable illegal ligatures
\fi

\begin{document}
\maketitle

\#\#Loading Libraries

\begin{Shaded}
\begin{Highlighting}[]
\NormalTok{knitr}\SpecialCharTok{::}\NormalTok{opts\_chunk}\SpecialCharTok{$}\FunctionTok{set}\NormalTok{(}\AttributeTok{echo=}\ConstantTok{TRUE}\NormalTok{, }\AttributeTok{cache=}\ConstantTok{TRUE}\NormalTok{, }\AttributeTok{fig.asp=}\FloatTok{0.65}\NormalTok{, }\AttributeTok{fig.width=}\FloatTok{5.5}\NormalTok{)}
\FunctionTok{library}\NormalTok{(ggplot2)}
\FunctionTok{library}\NormalTok{(tidyverse)}
\FunctionTok{theme\_set}\NormalTok{(}\FunctionTok{theme\_bw}\NormalTok{(}\AttributeTok{base\_size =} \DecValTok{14}\NormalTok{)) }\CommentTok{\# Tested others, put 14 since it is best for your pc}
\FunctionTok{head}\NormalTok{(mpg)}
\end{Highlighting}
\end{Shaded}

\begin{verbatim}
## # A tibble: 6 x 11
##   manufacturer model displ  year   cyl trans      drv     cty   hwy fl    class 
##   <chr>        <chr> <dbl> <int> <int> <chr>      <chr> <int> <int> <chr> <chr> 
## 1 audi         a4      1.8  1999     4 auto(l5)   f        18    29 p     compa~
## 2 audi         a4      1.8  1999     4 manual(m5) f        21    29 p     compa~
## 3 audi         a4      2    2008     4 manual(m6) f        20    31 p     compa~
## 4 audi         a4      2    2008     4 auto(av)   f        21    30 p     compa~
## 5 audi         a4      2.8  1999     6 auto(l5)   f        16    26 p     compa~
## 6 audi         a4      2.8  1999     6 manual(m5) f        18    26 p     compa~
\end{verbatim}

\begin{center}\rule{0.5\linewidth}{0.5pt}\end{center}

\hypertarget{exercises}{%
\subsubsection{3.2.4 Exercises}\label{exercises}}

\hypertarget{exercise-4-10-pts}{%
\paragraph{(1) 3.2.4 Exercise 4 (10 pts)}\label{exercise-4-10-pts}}

For the \texttt{mpg} dataset, make a scatterplot of \texttt{hwy} vs
\texttt{cyl}. What does this plot tell you about these two variables?

\begin{Shaded}
\begin{Highlighting}[]
\NormalTok{plot1 }\OtherTok{\textless{}{-}} \FunctionTok{ggplot}\NormalTok{(mpg, }\FunctionTok{aes}\NormalTok{(}\AttributeTok{x =}\NormalTok{ hwy, }\AttributeTok{y =}\NormalTok{ cyl, }\AttributeTok{color =}\NormalTok{ cyl)) }\SpecialCharTok{+}
  \FunctionTok{geom\_point}\NormalTok{()}

\NormalTok{plot1}
\end{Highlighting}
\end{Shaded}

\includegraphics{Homework-1.sa_files/figure-latex/unnamed-chunk-2-1.pdf}

\begin{Shaded}
\begin{Highlighting}[]
\CommentTok{\# From this scatter plot it would be reasonable to assume negative correlation between the amount of cylinders and the highway miles per gallon.}
\end{Highlighting}
\end{Shaded}

\hypertarget{exercise-5-10-pts}{%
\paragraph{(2) 3.2.4 Exercise 5 (10 pts)}\label{exercise-5-10-pts}}

What happens if you make a scatterplot of \texttt{class}
vs.~\texttt{drv}. Is this plot useful? Why or why not?

\begin{Shaded}
\begin{Highlighting}[]
\NormalTok{plot2 }\OtherTok{\textless{}{-}} \FunctionTok{ggplot}\NormalTok{(mpg, }\FunctionTok{aes}\NormalTok{(}\AttributeTok{x =}\NormalTok{ drv, }\AttributeTok{y =}\NormalTok{ class)) }\SpecialCharTok{+}
  \FunctionTok{geom\_point}\NormalTok{()}

\NormalTok{plot2}
\end{Highlighting}
\end{Shaded}

\includegraphics{Homework-1.sa_files/figure-latex/unnamed-chunk-3-1.pdf}

\begin{Shaded}
\begin{Highlighting}[]
\CommentTok{\# This plot can be useful when determining the all the possible different drive trains of each type of vehicle, but the information could be given in a better way other than a scatter plot.}
\end{Highlighting}
\end{Shaded}

\hypertarget{exercises-1}{%
\subsubsection{3.3.1 Exercises}\label{exercises-1}}

\hypertarget{exercise-2-10-pts}{%
\paragraph{(3) 3.3.1 Exercise 2 (10 pts)}\label{exercise-2-10-pts}}

Which variables in \texttt{mpg} are categorical? Which are continuous?
(Hint: type \texttt{?mpg} to read the documentation for the dataset).
How can you see this information when you run \texttt{mpg}?

\begin{Shaded}
\begin{Highlighting}[]
\CommentTok{\# The categorical variables are: manufacturer, model, cyl, trans, drv, fl, class}
\CommentTok{\# The continuous variables are: displ, year, cty, hwy}

\CommentTok{\# The function below lets us check the variable type, knowing that integers and numerical based data types are the only data type that allows continuous variables we can check those data types for all the continuous variables. Leaving all other variables to be discrete.}
\FunctionTok{str}\NormalTok{(mpg)}
\end{Highlighting}
\end{Shaded}

\begin{verbatim}
## tibble [234 x 11] (S3: tbl_df/tbl/data.frame)
##  $ manufacturer: chr [1:234] "audi" "audi" "audi" "audi" ...
##  $ model       : chr [1:234] "a4" "a4" "a4" "a4" ...
##  $ displ       : num [1:234] 1.8 1.8 2 2 2.8 2.8 3.1 1.8 1.8 2 ...
##  $ year        : int [1:234] 1999 1999 2008 2008 1999 1999 2008 1999 1999 2008 ...
##  $ cyl         : int [1:234] 4 4 4 4 6 6 6 4 4 4 ...
##  $ trans       : chr [1:234] "auto(l5)" "manual(m5)" "manual(m6)" "auto(av)" ...
##  $ drv         : chr [1:234] "f" "f" "f" "f" ...
##  $ cty         : int [1:234] 18 21 20 21 16 18 18 18 16 20 ...
##  $ hwy         : int [1:234] 29 29 31 30 26 26 27 26 25 28 ...
##  $ fl          : chr [1:234] "p" "p" "p" "p" ...
##  $ class       : chr [1:234] "compact" "compact" "compact" "compact" ...
\end{verbatim}

\hypertarget{exercise-3-10-pts}{%
\paragraph{(4) 3.3.1 Exercise 3 (10 pts)}\label{exercise-3-10-pts}}

Consider the scatterplot of displacement vs miles per gallon that we
have been studying:

\includegraphics{Homework-1.sa_files/figure-latex/unnamed-chunk-5-1.pdf}

Map a continuous variable to \texttt{color}, \texttt{size}, and
\texttt{shape}, and show the resulting graphs. How do these aesthetics
behave differently for categorical vs.~continuous variables?

\begin{Shaded}
\begin{Highlighting}[]
\NormalTok{color1 }\OtherTok{\textless{}{-}} \FunctionTok{ggplot}\NormalTok{(}\AttributeTok{data =}\NormalTok{ mpg) }\SpecialCharTok{+}
  \FunctionTok{geom\_point}\NormalTok{(}\AttributeTok{mapping =} \FunctionTok{aes}\NormalTok{(}\AttributeTok{x =}\NormalTok{ displ, }\AttributeTok{y =}\NormalTok{ hwy, }\AttributeTok{color =}\NormalTok{ cty))}

\NormalTok{size1 }\OtherTok{\textless{}{-}} \FunctionTok{ggplot}\NormalTok{(}\AttributeTok{data =}\NormalTok{ mpg) }\SpecialCharTok{+}
  \FunctionTok{geom\_point}\NormalTok{(}\AttributeTok{mapping =} \FunctionTok{aes}\NormalTok{(}\AttributeTok{x =}\NormalTok{ displ, }\AttributeTok{y =}\NormalTok{ hwy, }\AttributeTok{size =}\NormalTok{ cty))}

\NormalTok{shape1 }\OtherTok{\textless{}{-}} \FunctionTok{ggplot}\NormalTok{(}\AttributeTok{data =}\NormalTok{ mpg) }\SpecialCharTok{+}
  \FunctionTok{geom\_point}\NormalTok{(}\AttributeTok{mapping =} \FunctionTok{aes}\NormalTok{(}\AttributeTok{x =}\NormalTok{ displ, }\AttributeTok{y =}\NormalTok{ hwy, }\AttributeTok{shape =}\NormalTok{ cty))}


\NormalTok{color1 }\CommentTok{\#Color operates as more of a gradient of a single color rather than several              different colors}
\end{Highlighting}
\end{Shaded}

\includegraphics{Homework-1.sa_files/figure-latex/unnamed-chunk-6-1.pdf}

\begin{Shaded}
\begin{Highlighting}[]
\NormalTok{size1  }\CommentTok{\#Size fits the continuous variable more since the changes in size fit more               implies a relationship which discrete variables might not have.}
\end{Highlighting}
\end{Shaded}

\includegraphics{Homework-1.sa_files/figure-latex/unnamed-chunk-6-2.pdf}

\begin{Shaded}
\begin{Highlighting}[]
\CommentTok{\#shape1 \#Does not work, since a shape cannot be properly scaled for a continous variable}
\end{Highlighting}
\end{Shaded}

\hypertarget{exercise-6-10-pts}{%
\paragraph{(5) 3.3.1 Exercise 6 (10 pts)}\label{exercise-6-10-pts}}

What happens if you map an aesthetic to something other than a variable
name, like \texttt{aes(color\ =\ displ\ \textless{}\ 4)}? Note that you
will also need to specify \(x\) and \(y\).

\begin{Shaded}
\begin{Highlighting}[]
\FunctionTok{ggplot}\NormalTok{(}\AttributeTok{data =}\NormalTok{ mpg) }\SpecialCharTok{+}
  \FunctionTok{geom\_point}\NormalTok{(}\AttributeTok{mapping =} \FunctionTok{aes}\NormalTok{(}\AttributeTok{x =}\NormalTok{ displ, }\AttributeTok{y =}\NormalTok{ hwy, }\AttributeTok{color =}\NormalTok{ displ }\SpecialCharTok{\textless{}} \DecValTok{4}\NormalTok{))}
\end{Highlighting}
\end{Shaded}

\includegraphics{Homework-1.sa_files/figure-latex/unnamed-chunk-7-1.pdf}

\begin{Shaded}
\begin{Highlighting}[]
\CommentTok{\# Two separate colors will represent the results, either true or false.}
\end{Highlighting}
\end{Shaded}

\hypertarget{exercises-2}{%
\subsubsection{3.5.1 Exercises}\label{exercises-2}}

\hypertarget{exercise-1-5-pts}{%
\paragraph{(6) 3.5.1 Exercise 1 (5 pts)}\label{exercise-1-5-pts}}

What happens if you facet on a continuous variable? Show an example.
Discuss when this might be useful, and when it is likely not useful.

\begin{Shaded}
\begin{Highlighting}[]
\FunctionTok{ggplot}\NormalTok{(}\AttributeTok{data =}\NormalTok{ mpg) }\SpecialCharTok{+} 
  \FunctionTok{geom\_point}\NormalTok{(}\AttributeTok{mapping =} \FunctionTok{aes}\NormalTok{(}\AttributeTok{x =}\NormalTok{ displ, }\AttributeTok{y =}\NormalTok{ hwy, }\AttributeTok{color =}\NormalTok{ class)) }\SpecialCharTok{+}
  \FunctionTok{facet\_wrap}\NormalTok{(}\AttributeTok{facets =} \FunctionTok{vars}\NormalTok{(cty))}
\end{Highlighting}
\end{Shaded}

\includegraphics{Homework-1.sa_files/figure-latex/unnamed-chunk-8-1.pdf}

\begin{Shaded}
\begin{Highlighting}[]
\CommentTok{\# Several facets pop up for every option, this can be useful when a researcher wants to   deep dive on the effect a continuous variable has. However, it is not very useful outside of that since it is a lot of information and hard to digest. Plus it will have a hard time loading all that data in a readable fashion.}
\end{Highlighting}
\end{Shaded}

\hypertarget{exercise-4-10-pts-1}{%
\paragraph{(7) 3.5.1 Exercise 4 (10 pts)}\label{exercise-4-10-pts-1}}

Consider the first faceted plot in this section:

\begin{Shaded}
\begin{Highlighting}[]
\FunctionTok{ggplot}\NormalTok{(}\AttributeTok{data =}\NormalTok{ mpg) }\SpecialCharTok{+} 
  \FunctionTok{geom\_point}\NormalTok{(}\AttributeTok{mapping =} \FunctionTok{aes}\NormalTok{(}\AttributeTok{x =}\NormalTok{ displ, }\AttributeTok{y =}\NormalTok{ hwy)) }\SpecialCharTok{+} 
  \FunctionTok{facet\_wrap}\NormalTok{(}\AttributeTok{facets =} \FunctionTok{vars}\NormalTok{(class), }\AttributeTok{nrow =} \DecValTok{2}\NormalTok{)}
\end{Highlighting}
\end{Shaded}

\includegraphics{Homework-1.sa_files/figure-latex/unnamed-chunk-9-1.pdf}

Create a comparable graph that uses the color aesthetic. What are the
advantages to using faceting instead of the color aesthetic? What are
the disadvantages? How might the balance change if you had a larger
dataset?

\begin{Shaded}
\begin{Highlighting}[]
\FunctionTok{ggplot}\NormalTok{(}\AttributeTok{data =}\NormalTok{ mpg) }\SpecialCharTok{+} 
  \FunctionTok{geom\_point}\NormalTok{(}\AttributeTok{mapping =} \FunctionTok{aes}\NormalTok{(}\AttributeTok{x =}\NormalTok{ displ, }\AttributeTok{y =}\NormalTok{ hwy, }\AttributeTok{color =}\NormalTok{ class))}
\end{Highlighting}
\end{Shaded}

\includegraphics{Homework-1.sa_files/figure-latex/unnamed-chunk-10-1.pdf}

\begin{Shaded}
\begin{Highlighting}[]
\CommentTok{\# The advantage of using faceting is that it is much easier to compare how each class interacts with the given variables. The disadvantages would come around if a researcher is comparing each class to itself over the given variables. Colors is more advantageous the more unique classes in the variable since it is easier to compare, despite losing the ability to clearly see patterns.}
\end{Highlighting}
\end{Shaded}

\hypertarget{exercises-3}{%
\subsubsection{3.6.1 Exercises}\label{exercises-3}}

\hypertarget{exercises-2-10-pts}{%
\paragraph{(8) 3.6.1 Exercises 2 (10 pts)}\label{exercises-2-10-pts}}

Run this code in your head and predict what the output will look like.
Then, run the code in R and check your predictions, and discuss the
result.

\begin{Shaded}
\begin{Highlighting}[]
\FunctionTok{ggplot}\NormalTok{(}\AttributeTok{data =}\NormalTok{ mpg, }\AttributeTok{mapping =} \FunctionTok{aes}\NormalTok{(}\AttributeTok{x =}\NormalTok{ displ, }\AttributeTok{y =}\NormalTok{ hwy, }\AttributeTok{color =}\NormalTok{ drv)) }\SpecialCharTok{+} 
  \FunctionTok{geom\_point}\NormalTok{() }\SpecialCharTok{+} 
  \FunctionTok{geom\_smooth}\NormalTok{(}\AttributeTok{se =} \ConstantTok{FALSE}\NormalTok{)}
\end{Highlighting}
\end{Shaded}

\includegraphics{Homework-1.sa_files/figure-latex/unnamed-chunk-11-1.pdf}

\begin{Shaded}
\begin{Highlighting}[]
\CommentTok{\# Predictions, a scatter plot and a line graph will appear in the same space and will have the same variables and colors. }

\CommentTok{\# My prediction proved to be true, since each "geom" line of code states the type of graph, each graph will follow the same parameters labeled above.}
\end{Highlighting}
\end{Shaded}

\hypertarget{exercises-4-5-pts}{%
\paragraph{(9) 3.6.1 Exercises 4 (5 pts)}\label{exercises-4-5-pts}}

What does the \texttt{se} argument to \texttt{geom\_smooth()} do?

\begin{Shaded}
\begin{Highlighting}[]
\CommentTok{\# It displays the confidence interval around the machine learning technique used to calculate the smooth line. }
\end{Highlighting}
\end{Shaded}

\hypertarget{exercises-5-5-pts}{%
\paragraph{(10) 3.6.1 Exercises 5 (5 pts)}\label{exercises-5-5-pts}}

Will these two graphs look different? Why or why not?

\begin{Shaded}
\begin{Highlighting}[]
\FunctionTok{ggplot}\NormalTok{(}\AttributeTok{data =}\NormalTok{ mpg, }\AttributeTok{mapping =} \FunctionTok{aes}\NormalTok{(}\AttributeTok{x =}\NormalTok{ displ, }\AttributeTok{y =}\NormalTok{ hwy)) }\SpecialCharTok{+} 
  \FunctionTok{geom\_point}\NormalTok{() }\SpecialCharTok{+} 
  \FunctionTok{geom\_smooth}\NormalTok{()}

\FunctionTok{ggplot}\NormalTok{() }\SpecialCharTok{+} 
  \FunctionTok{geom\_point}\NormalTok{(}\AttributeTok{data =}\NormalTok{ mpg, }\AttributeTok{mapping =} \FunctionTok{aes}\NormalTok{(}\AttributeTok{x =}\NormalTok{ displ, }\AttributeTok{y =}\NormalTok{ hwy)) }\SpecialCharTok{+} 
  \FunctionTok{geom\_smooth}\NormalTok{(}\AttributeTok{data =}\NormalTok{ mpg, }\AttributeTok{mapping =} \FunctionTok{aes}\NormalTok{(}\AttributeTok{x =}\NormalTok{ displ, }\AttributeTok{y =}\NormalTok{ hwy))}

\CommentTok{\# No since both are saying the same thing. However, the second one is not allowed in any of the 50 states since it rewrites the exact same labeling twice.}
\end{Highlighting}
\end{Shaded}

\hypertarget{exercises-6}{%
\paragraph{3.6.1 Exercises 6}\label{exercises-6}}

Recreate the R code needed to generate each of these graphs:

\hypertarget{pts}{%
\subparagraph{(11) (6.1, 5 pts)}\label{pts}}

\begin{Shaded}
\begin{Highlighting}[]
\FunctionTok{ggplot}\NormalTok{(}\AttributeTok{data =}\NormalTok{ mpg, }\AttributeTok{mapping =} \FunctionTok{aes}\NormalTok{(}\AttributeTok{x =}\NormalTok{ displ, }\AttributeTok{y =}\NormalTok{ hwy)) }\SpecialCharTok{+} 
  \FunctionTok{geom\_point}\NormalTok{() }\SpecialCharTok{+} 
  \FunctionTok{geom\_smooth}\NormalTok{(}\AttributeTok{se =} \ConstantTok{FALSE}\NormalTok{)}
\end{Highlighting}
\end{Shaded}

\begin{verbatim}
## `geom_smooth()` using method = 'loess' and formula 'y ~ x'
\end{verbatim}

\includegraphics{Homework-1.sa_files/figure-latex/unnamed-chunk-14-1.pdf}

\hypertarget{pts-1}{%
\subparagraph{(12) (6.2, 5 pts)}\label{pts-1}}

\begin{Shaded}
\begin{Highlighting}[]
\FunctionTok{ggplot}\NormalTok{(}\AttributeTok{data =}\NormalTok{ mpg, }\AttributeTok{mapping =} \FunctionTok{aes}\NormalTok{(}\AttributeTok{x =}\NormalTok{ displ, }\AttributeTok{y =}\NormalTok{ hwy)) }\SpecialCharTok{+} 
  \FunctionTok{geom\_point}\NormalTok{() }\SpecialCharTok{+} 
  \FunctionTok{geom\_smooth}\NormalTok{(}\FunctionTok{aes}\NormalTok{(}\AttributeTok{color =}\NormalTok{ drv), }\AttributeTok{se =} \ConstantTok{FALSE}\NormalTok{) }
\end{Highlighting}
\end{Shaded}

\begin{verbatim}
## `geom_smooth()` using method = 'loess' and formula 'y ~ x'
\end{verbatim}

\includegraphics{Homework-1.sa_files/figure-latex/unnamed-chunk-15-1.pdf}

\begin{Shaded}
\begin{Highlighting}[]
\CommentTok{\# Could not find an aes function that would split the data based on a variable without using color}
\end{Highlighting}
\end{Shaded}

\hypertarget{pts-2}{%
\subparagraph{(13) (6.3, 5 pts)}\label{pts-2}}

\begin{Shaded}
\begin{Highlighting}[]
\FunctionTok{ggplot}\NormalTok{(}\AttributeTok{data =}\NormalTok{ mpg, }\AttributeTok{mapping =} \FunctionTok{aes}\NormalTok{(}\AttributeTok{x =}\NormalTok{ displ, }\AttributeTok{y =}\NormalTok{ hwy, }\AttributeTok{color =}\NormalTok{ drv)) }\SpecialCharTok{+} 
  \FunctionTok{geom\_point}\NormalTok{() }\SpecialCharTok{+} 
  \FunctionTok{geom\_smooth}\NormalTok{(}\AttributeTok{se =} \ConstantTok{FALSE}\NormalTok{)}
\end{Highlighting}
\end{Shaded}

\begin{verbatim}
## `geom_smooth()` using method = 'loess' and formula 'y ~ x'
\end{verbatim}

\includegraphics{Homework-1.sa_files/figure-latex/unnamed-chunk-16-1.pdf}

\hypertarget{pts-3}{%
\subparagraph{(14) (6.4, 5 pts)}\label{pts-3}}

\begin{Shaded}
\begin{Highlighting}[]
\FunctionTok{ggplot}\NormalTok{(}\AttributeTok{data =}\NormalTok{ mpg, }\AttributeTok{mapping =} \FunctionTok{aes}\NormalTok{(}\AttributeTok{x =}\NormalTok{ displ, }\AttributeTok{y =}\NormalTok{ hwy)) }\SpecialCharTok{+} 
  \FunctionTok{geom\_point}\NormalTok{(}\FunctionTok{aes}\NormalTok{(}\AttributeTok{color =}\NormalTok{ drv)) }\SpecialCharTok{+} 
  \FunctionTok{geom\_smooth}\NormalTok{(}\AttributeTok{se =} \ConstantTok{FALSE}\NormalTok{)}
\end{Highlighting}
\end{Shaded}

\begin{verbatim}
## `geom_smooth()` using method = 'loess' and formula 'y ~ x'
\end{verbatim}

\includegraphics{Homework-1.sa_files/figure-latex/unnamed-chunk-17-1.pdf}

\hypertarget{pts-4}{%
\subparagraph{(15) (6.5, 5 pts)}\label{pts-4}}

\begin{Shaded}
\begin{Highlighting}[]
\FunctionTok{ggplot}\NormalTok{(}\AttributeTok{data =}\NormalTok{ mpg, }\AttributeTok{mapping =} \FunctionTok{aes}\NormalTok{(}\AttributeTok{x =}\NormalTok{ displ, }\AttributeTok{y =}\NormalTok{ hwy)) }\SpecialCharTok{+} 
  \FunctionTok{geom\_point}\NormalTok{(}\FunctionTok{aes}\NormalTok{(}\AttributeTok{color =}\NormalTok{ drv)) }\SpecialCharTok{+} 
  \FunctionTok{geom\_smooth}\NormalTok{(}\FunctionTok{aes}\NormalTok{(}\AttributeTok{linetype =}\NormalTok{ drv),}\AttributeTok{se =} \ConstantTok{FALSE}\NormalTok{)}
\end{Highlighting}
\end{Shaded}

\begin{verbatim}
## `geom_smooth()` using method = 'loess' and formula 'y ~ x'
\end{verbatim}

\includegraphics{Homework-1.sa_files/figure-latex/unnamed-chunk-18-1.pdf}

\hypertarget{pts-5}{%
\subparagraph{(16) (6.6, 5 pts)}\label{pts-5}}

\begin{Shaded}
\begin{Highlighting}[]
\FunctionTok{ggplot}\NormalTok{(}\AttributeTok{data =}\NormalTok{ mpg, }\AttributeTok{mapping =} \FunctionTok{aes}\NormalTok{(}\AttributeTok{x =}\NormalTok{ displ, }\AttributeTok{y =}\NormalTok{ hwy)) }\SpecialCharTok{+} 
  \FunctionTok{geom\_point}\NormalTok{(}\AttributeTok{size =} \FloatTok{3.5}\NormalTok{, }\AttributeTok{color =} \StringTok{"grey"}\NormalTok{) }\SpecialCharTok{+} \CommentTok{\#White was difficult to see}
  \FunctionTok{geom\_point}\NormalTok{(}\FunctionTok{aes}\NormalTok{(}\AttributeTok{color =}\NormalTok{ drv))}
\end{Highlighting}
\end{Shaded}

\includegraphics{Homework-1.sa_files/figure-latex/unnamed-chunk-19-1.pdf}

\end{document}
